\documentclass[11pt]{article}

\usepackage{spikey}
\usepackage{amssymb}
\usepackage[margin=1in]{geometry}

\usepackage{setspace}
\linespread{1.15}

\newcommand{\s}[0]{$\sigma$}

\title{Notes on Measure Theory}
\author{Tianyu Du}

\begin{document}
	\maketitle
	
	\section{Sigma Algebra}
	\begin{definition}
		For a set $X$, a set $\mc{A} \subseteq \mc{P}(X)$ is a \textbf{\s-algebra} if it satisfies the following properties:
		\begin{enumerate}
			\item $\varnothing, X \in \mc{A}$;
			\item for all $A \in \mc{A}$, $A^c \in \mc{A}$ as well;
			\item for a sequence in $\mc{A}$, $\{A_i\}_{i\in\N}$, the union $\bigcup_{i\in\N} \in \mc{A}$ as well.
		\end{enumerate}
		An element $A \in \mc{A}$ is called a \textbf{$\mc{A}$-measurable set}.
	\end{definition}
	
	\begin{remark}
		It's easy to show that the largest \s-algebra of set $X$ is the power set $\mc{P}(X)$, and the smallest \s-algebra is $\{\varnothing, X\}$.
	\end{remark}
	
	\begin{theorem}
		Let $\{\mc{A}_i\}_{i \in I}$ be the collection of all $\sigma$-algebra on $X$. Then, $\bigcap_{i \in I} \mc{A}_i$ is also a $\sigma$-algebra on $X$.
	\end{theorem}
	
	\begin{proof}
		
	\end{proof}
	
	\begin{definition}
		For $\mc{M} \subseteq \mc{P}(X)$ ($\mc{A}$ is not necessarily a $\sigma$-algebra), the smallest $\sigma$-algebra (by taking intersections) containing $\mc{M}$ is defined as the \textbf{$\sigma$-algebra generated by $\mc{M}$}.
		The generated $\sigma$-algebra is simply the intersection of all $\sigma$-algebras that are supersets of $\mc{M}$.
		\begin{align}
			\sigma(\mc{M}) = \bigcap_{\mc{A} \supseteq \mc{M}\ s.t.\ \mc{A}\tx{ is $\sigma$-algebra}} \mc{A}
		\end{align}
	\end{definition}
	
	\begin{definition}
		Let $(X, \tau)$ be a topological space, then the \textbf{Borel algebra} is $\sigma$-algebra generated by the collection of open sets $\tau$.
		\begin{align}
			\mc{B}(X) := \sigma(\tau)
		\end{align}
	\end{definition}
	
	\begin{remark}
		We do not use the entire power set for analysis because it's too large to construct a sensible measure on (see Theorem \ref{thm:nomeasure}).
	\end{remark}
	
	\begin{theorem} \label{thm:nomeasure}
		There is no measure $\mu$ on $(\R, \mc{P}(\R))$ satisfying the following two conditions: (i) $\mu((a, b]) = b-a$ for every $a < b$ and (ii) $\mu(x+A) = \mu(A)$ for all $a \in \R$ and $A \in \mc{P}(\R)$. 
	\end{theorem}
	
	\begin{proof}
		Suppose, for contradiction, there exists such a measure $\mu$, then $\mu((0, 1]) = 1 < \infty$.
		
		Claim: the only measure on $(\R, \mc{P}(\R))$ satisfying $\mu((0, 1]) < \infty$ and $\mu(x+A) = \mu(A)$ is the zero measure.
		
		To prove the claim, let $I := (0, 1]$ and defien the following equivalence relation on $I$:
		\begin{align}
			x \sim y \iff x - y \in \Q
		\end{align}
		the corresponding equivalence class of $x$ on $I$ can be written as
		\begin{align}
			[x] = \{x+r: r \in \Q \land x+r \in I\}
		\end{align}
		The collection of all such equivalence classes, $\mc{A}$, is a disjoint decomposition of $I$. (for every $x \in I$, $[x]$ must in $\mc{A}$ and $x \in [x]$ trivially. If there exists different $[x] \neq [y]$ but $[x] \cap [y] \neq \varnothing$, take $z \in [x] \cap [y]$, by the transitivity of equivalence relation, $x \sim z \sim y$. Therefore, $[x] = [y]$, contradiction.)
		
		For each $[x] \in \mc{A}$, take exactly one $a_x \in [x]$ and define set $A := \{a_x: [x] \in \mc{A}\}$. As a result, set $A$ satisfies the following two properties:
		\begin{enumerate}
			\item $\forall x \in I, \exists a_x \in A\ s.t.\ a_x \in [x]$.
			\item $\forall x, y \in A$, $x \sim y \implies x = y$.
		\end{enumerate}
		Since $\Q \cup (-1, 1]$ is countable, let $(r_n)_{n \in \N}$ be an enumeration of all elements in it.
		
		For each $n \in \N$, define $A_n := r_n + A$.
		
		Note that for any $m, n$ such that $A_m \cap A_n \neq \varnothing$, take $x \in A_m \cap A_n$. By definition, 
		\begin{align}
				x &= r_n + a_n \\
				x &= r_m + a_m
		\end{align}
		where $a_n, a_m \in A$ and $r_n, r_m \in \Q$. Consequently,
		\begin{align}
			a_n - a_m = r_m - r_n \in \Q
		\end{align}
		Therefore, $a_n \sim a_m$. By the second property of $A$, $a_n = a_m$. Thus, $r_m = r_n$ and $m = n$.
		
		Take the counterposition of what we just proved, $m \neq n \implies A_m \cap A_n = \varnothing$.
		
		Let $z \in (0, 1]$, there exists some $a \in A$ such that $z \in [x]$. That is, $z = x + r$ for some $r \in \Q \cap (-1, 1]$. There must exist some $m \in \N$ such that $r_m = r$, and consequently, $z \in A_m$.
		 
		Therefore, $(0, 1] \subseteq \bigcup_{n\in \N} A_n \subseteq (-1, 2]$ (the second relation is obvious). Moreover,
		\begin{align}
			\mu((0, 1]) \leq \mu(\bigcup_{n\in \N} A_n) \leq \mu((-1, 2]) = \mu((-1, 0]) + \mu((0, 1]) + \mu((1, 2]) = 3 \mu((0, 1])
		\end{align}
		Note that we just proved $\bigcup_{n\in \N} A_n$ is a disjoint union, hence,
		\begin{align}
			\mu((0, 1]) \leq \sum_{n=1}^\infty \mu(A_n) \leq 3\mu((0, 1]) \\
			\implies((0, 1]) \mu \leq \sum_{n=1}^\infty \mu(A + r_n) \leq 3\mu((0, 1]) \\
			\implies \mu((0, 1]) \leq \sum_{n=1}^\infty \mu(A) \leq 3\mu((0, 1])
		\end{align}
		Since $\mu((0, 1])$ is finite, the only value $\mu(A)$ can take is zero, and $\mu(I) = 0$ as well. Consequently, for any set $S \in \mc{P}(\R)$, if $S \subseteq I$, then $\mu(S) \leq \mu(I)$ and $\mu(S) = 0$. Otherwise, let $l = \floor{\inf(S)}$ and $u = \ceil{\sup(S)}$.
		\begin{align}
			I \subseteq S \subseteq \bigcup_{n=l}^{u} (n, n+1]
		\end{align}
		Therefore,
		\begin{align}
			0 \leq \mu(S) \leq \sum_{n=l}^{u} \mu(n + (0, 1]) = \sum_{n=l}^{u} \mu((0, 1]) = 0
		\end{align}
		It's shown that $\mu(S) = 0$ for every $S \subseteq \mc{P}(\R)$.
		
		This leads to a contradiction to the first property required ($\mu((a, b]) = b-a$).
	\end{proof}
	
	\section{Measurable Spaces and Measurable Maps}
	\begin{definition}
		Let $(X_1, \mc{A}_1)$ and $(X_2, \mc{A}_2)$ be two measurable spaces. A function $f: X_1 \to X_2$ is a \textbf{measurable map} with respect to $\mc{A}_1$ and $\mc{A}_2$ if
		\begin{align}
			f^{-1}(A_2) \in \mc{A}_1\quad \forall A_2 \in \mc{A}_2
		\end{align}
		That is, the pre-image of every set in $\mc{A}_2$ is an element in $\mc{A}_1$ as well.
	\end{definition}
	
	\begin{theorem}
		Let $(\Omega, \mc{A})$ be a measurable space, then the indicator function for any $A \in \mc{A}$, $\mc{X}_A: \Omega \to \R$, is measurable with respect to $\mc{A}$ and $\mc{B}(\R)$.
		\begin{align}
			\mc{X}_A(\omega) := \begin{cases}
				1 &\tx{ if } \omega \in A \\
				0 &\tx{ otherwise}
			\end{cases}
		\end{align}
	\end{theorem}
	
	\begin{proof}
		
	\end{proof}
	
	\begin{theorem}
		The composition of measurable maps is measurable.
	\end{theorem}
	
	\begin{proof}
		
	\end{proof}
	
	\begin{theorem}
		For measurable spaces $(X, \mc{A})$ and $(\R, \mc{B}(\R)$ and measurable maps $f, g: \Omega \to \R$, $f + g$, $f - g$ and $\abs{f}$ are measurable.
	\end{theorem}
	
	\begin{proof}
		
	\end{proof}
	
	\section{Lebesgue Measures}
\end{document}















