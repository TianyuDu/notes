\documentclass[11pt]{article}

\usepackage{spikey}
\usepackage{amssymb}
\usepackage[margin=1in]{geometry}

\usepackage{setspace}
\linespread{1.15}

\title{MATH205A: Measure Theory}

\newcommand{\salg}[0]{$\sigma$-algebra}
\newcommand{\br}[0]{$\mc{B}(\R)$}

\begin{document}
	\maketitle
	\section{Lecture 1}
	\subsection{Motivation}
	\textbf{Motivation of this course} is to define a notion of \emph{length} on subsets of $\R$ such that
	\begin{enumerate}
		\item $length([a,b]) = b - a$.
		\item (countable additivity) $length(\bigcup^\infty A_i) = \sum^\infty length(A_i)$ where $A_i$'s are disjoint.
		\item (translation invariance) for all $a \in \R$, $length(A + a) = length(A)$.
	\end{enumerate}
	\begin{fact} it is impossible to construct such length for all subsets of $\R$.
	\begin{proof}
		This proof shows it is impossible to construct a notion of length on $[0, 1]$ with desired properties.
		
		For $x, y \in [0, 1]$, define an equivalence relation as $x \sim y \iff x - y \in \Q$. By the axiom of choice, we may construct a set $A$ containing exactly one element from each equivalence class of $x \in [0, 1]$. Obviously, $A \subseteq [0, 1]$.
		
		For each $r \in [-1, 1] \cap \Q$, let $A_r := A + r$, and $A_r \subseteq [-1, 2]$.
		By translation invariance, $length(A_r) = length(A)$.
		Note that for any $y \in [0, 1]$, there exists some $x \in A$ such that $x \sim y$, therefore, $y \in A_{y-x} \subseteq \bigcup_r A_r$. Hence, $[0, 1] \subseteq \bigcup_r A_r$.
		
		If the notion of length satisfies countable additivity, $length(\bigcup_r A_r)$ is either zero or infinity, which leads to a contradiction.
	\end{proof}
	\end{fact}
	
	\textbf{Lebesgue's Resolution}: we only defines length for a subset of $\mc{P}(\R)$, which contains \emph{everything that may ever arrive in practice}, i.e., $\sigma$-algebras.
	
	\subsection{Algebras and \salg}
	
	\begin{definition}
		Let $X$ be a set, a collection $\mc{A}$ of subsets of $X$ is called an \textbf{algebra} if
		\begin{enumerate}
			\item $X \in \mc{A}$,
			\item $A \in \mc{A} \implies A^c \in \mc{A}$,
			\item $A, B \in \mc{A} \implies A \cup B \in \mc{A}$.
		\end{enumerate}
		Consequently: (1) $A, B \in \mc{A} \implies A \cap B \in \mc{A}$; (2) $A_1, \dots, A_n \in \mc{A} \implies \bigcup_i A_i, \bigcap_i A_i \in \mc{A}$ (easily shown by induction); (3) $\varnothing \in \mc{A}$.
	\end{definition}
	
	\begin{definition}
		Let $X$ be a set, a collection $\mc{A}$ of subsets of $X$ is called a $\sigma$-\textbf{algebra} if
		\begin{enumerate}
			\item $X \in \mc{A}$,
			\item $A \in \mc{A} \implies A^c \in \mc{A}$,
			\item $A_1, A_2 \dots, \in \mc{A}, \implies \bigcup_i^\infty A_i \in \mc{A}$.
		\end{enumerate}
	\end{definition}
	
	\begin{example}[trivial examples]
		The power set of $X$ is a $\sigma$-algebra on $X$; $\{\varnothing, X\}$ is a $\sigma$-algebra on $X$.
	\end{example}
	
	\begin{example}[finite/co-finite algebra]
		Let $X$ be an infinite set and $\mc{A}$ be the collection of subsets $A$ such that either $A$ is finite or $A^c$ is finite. $\mc{A}$ is an algebra.
		\begin{proof}
			$X \in \mc{A}$ since $X^c = \varnothing$ is finite. For a $X \in \mc{A}$, if $X$ is finite, then $X^c \in \mc{A}$. If $X$ is infinite, $X^c$ is finite and $X^c \in \mc{A}$. Let $A, B \in \mc{A}$, if both $A$ and $B$ are finite, $A \cup B$ is finite and in $\mc{A}$. If $A$ is finite and $B$ is co-finite, then $(A \cup B)^c = A^c \cap B^c \subseteq B^c$ is finite. If both $A$ and $B$ are co-finite, $(A \cup B)^c$ is finite so that $A \cup B \in \mc{A}$.
		\end{proof}
		Note the $\mc{A}$ is \ul{not} a $\sigma$-algebra if $X$ is infinite: take distinct points $x_1, x_2, \dots \in \mc{A}$, then the union of them is neither finite or co-finite, and therefore not in $\mc{A}$.
	\end{example}
	
	\begin{example}[countable/co-countable $\sigma$-algebra]
		The collection of subsets $A \subseteq X$, such that either $A$ is countable or $A^c$ is countable, forms a $\sigma$-algebra.
	\end{example}
	
	\begin{example}
		Let $X = \R$ and $\mc{A}$ be the collection of all \ul{finite} \ul{disjoint} unions of half-open intervals (i.e., sets like $(a, b], (-\infty, b], (a, \infty)$), $\mc{A}$ is an algebra. (Not working for open intervals).
	\end{example}
	
	\begin{example}[counter example]
		Let $X$ be an infinite set, $\mc{A}$ be the collection of finite subsets of $X$. Then, $\mc{A}$ is \ul{not} an algebra.
	\end{example}
	
	\begin{proposition}
		Let $X$ be a set and $\{\mc{A}_i\}_{i \in \mc{I}}$ be an arbitrary (not necessarily countable) collection of $\sigma$-algebras, then $\bigcap_{i \in \mc{I}} \mc{A}_i$ is a $\sigma$-algebra.
		\begin{proof}
			Since $X \in \mc{A}_i$ for all $i \in \mc{I}$
		\end{proof}
	\end{proposition}
	
	\begin{corollary}
		Let $X$ be a set, and $\mc{P}$ is an arbitrary collection of subsets of $X$, then $\exists !$ smallest $\sigma$-algebra $\mc{A}$ containing $\mc{P}$. That is, for any $\sigma$-algebra $\mc{B} \supseteq \mc{P}$, $\mc{A} \subseteq \mc{B}$. $\mc{A}$ is defined as the $\sigma$-algebra \textbf{generated by} $\mc{P}$, denoted as $\sigma(\mc{P})$.
	\end{corollary}
	
	\begin{proof}
		For any $\mc{P}$, the power set of $X$ is obviously a $\sigma$-algebra containing $\mc{P}$. Then we can take $\mc{A}$ as the intersection of all $\sigma$-algebras containing $\mc{P}$.
	\end{proof}
	
	\subsection{Borel $\sigma$-algebra}
	\begin{definition}
		The \textbf{Borel \salg} of $\R$, denoted as $\mc{B}(\R)$, is the \salg generated by the set of \ul{open intervals} in $\R$.
	\end{definition}
	
	\begin{fact}
		$\mc{B}(\R)$ is generated by the collection of all closed intervals as well.
		\begin{proof}
			Let $\mc{F}$ denote the \salg generated by all closed intervals. Any open interval can be written as a countable union of closed sets: $(a, b) = \bigcup_{n=1}^\infty [a+1/n, b-1/n]$, therefore $(a, b) \in \mc{F}$ and $\mc{B}(\R) \subseteq \mc{F}$.
			
			Similarly, $[a, b] = \bigcap_{n=1}^\infty (a-1/n, b+1/n)$, hence $\mc{B}(\R)$ is a \salg contains all closed sets. Therefore, $\mc{F} \subseteq \mc{B}(\R)$.
		\end{proof}
	\end{fact}
	
	\begin{fact}
		$\mc{B}(\R)$ is generated by \begin{enumerate}
			\item all open sets,
			\item all closed sets,
			\item all half-open intervals.
		\end{enumerate}
	\end{fact}
	
	\begin{example}[counter example]
		\br is \ul{not} generated by the collection of singletons.
		\begin{proof}
			
		\end{proof}
	\end{example}
	
	\begin{definition}
		The Borel algebra of $\R^d$, $\mc{B}(\R^d)$, is the \salg generated by
		\begin{enumerate}
			\item all open sets in $\R^d$,
			\item all closed sets in $\R^d$,
			\item all closed cubes (regions) in $\R^d$: $\prod_{i=1}^d [a_i, b_i]$.
		\end{enumerate}
	\end{definition}
	
	\subsection{Measures}
	\begin{definition}
		For a set $X$ and a \salg $\mc{A}$ of $X$, the pair $(X, \mc{A})$ is called a \textbf{measurable space}.
	\end{definition}
	
	\begin{definition}
		A \textbf{measure} $\mu$ on a measurable space $(X, \mc{A})$ is a map $\mu: \mc{A} \to [0, \infty]$ such that
		\begin{enumerate}
			\item $\mu(\varnothing) = 0$,
			\item $\mu(\cup_i^\infty A_i) = \sum_{i}^\infty \mu(A_i)$ for disjoint sequence $(A_i)$
		\end{enumerate}
		For now, we don't require the translation invariance property.
		
		The triple $(X, \mc{A}, \mu)$ is called a \textbf{measure space}.
	\end{definition}
	
	\begin{example}[counting measure]
		
	\end{example}
	
	\begin{example}[point-mass measure]
		
	\end{example}
	
	\begin{proposition}
		A measure $\mu$ possesses the following basic properties:
		\begin{enumerate}
			\item (Monotonicity) $A \subseteq B \implies \mu(A) \leq \mu(B)$.
			\item (Sub-additivity) $\mu(\bigcup_{i=1}^\infty A_i) \leq \sum_{i=1}^\infty \mu(A_i)$.
		\end{enumerate}
		\begin{proof}
			
		\end{proof}
	\end{proposition}
\end{document}




























